\chapter{Signal Conditioning and Processing}
\label{chap:\currfilebase}

In an ideal world, the signal output of a sensor would correlate to the measurand exactly. In real systems this is not the case because of a variety of reasons. In low-frequency applications, the most important ones are:

\begin{itemize}
    \item The voltage or current rating at a sensors output is not perfectly linear with respect to the measurand. Often the output is pseudo-linear in a limited range of values and deviates from the trajectory for values outside of this range.
    \item Noise and shifts introduced through the inherent impedances of analog components lead to deviations from the voltage or current rating of the sensor as well as deviations of these ratings with respect to the measurand itself.
    \item The quantization process causes the captured value space to have a finite resolution.
    \item Analogue signals can only be digitized with a finite sampling rate. A discrete set of data points is captured instead of a continuous signal.
\end{itemize}

The field of signal processing includes analysing, modifying and synthesizing signals. Most prominently, in data acquisition system we convert analog signals to digital ones that can be further processed without the parasitic effects of the analog realm. On the opposite side when addressing these parasitic effects one needs to apply signal conditioning. In other words, before every processing step of an analog signal we need to consider signal conditioning. When dealing with digital signals, no signal conditioning is required.

\section{Signal Conditioning}


\subsection{Excitation}

\subsection{Amplification}

\subsection{Filtering}