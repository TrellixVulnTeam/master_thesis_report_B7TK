\chapter*{Abstract}
%Context, Content and Conclusion summarized to 1 page.
% English version:

Experimental modal analysis in tandem with the modal model of a machine tool is a powerful tool for the evaluation of the machine tools' dynamic behavior. But because experimental modal analysis is an expensive procedure, both due to high instrument costs and the need for experienced operators, the modal model is often not verified.

With the aim to decrease instrument costs and increase the use of experimental modal analysis, the system presented in this thesis consists of a micro controller based data acquisition system, a modal impact hammer and a low-cost accelerometer. The latter is a capacitive micro-electro-mechanical-sensor and the impact hammer is using a strain gauge load cell as impact sensor. 

For the implementation of a low-cost modal analysis system based on the aforementioned components a micro controller based bus system and a specialized communication protocol is suggested.

%------------------------------------------
\cleardoublepage
\chapter*{Zusammenfassung}

Die experimentelle Modalanalyse, in Kombination mit dem modalen Modell einer Werkzeugmaschine, ist ein mächtiges Werkzeug, um das dynamische Verhalten der Werkzeugmaschine zu evaluieren. Teure Messinstrumente und der Bedarf an erfahrenen Bedienern machen die experimentelle Modalanalyse allerdings zu einem kostspieligen Unterfangen. Oft wird daher das modale Modell gar nicht validiert.

Mit dem Ziel die Kosten für Messinstrumente zu senken und den Gebrauch von experimenteller Modalanalyse zu steigern, wird in dieser Arbeit ein System vorgestellt, welches aus einem Mikrokontroller-basierten Datenakquisitionssystem, einem Impulshammer und einem kostengünstigen Beschleunigungssensor besteht. Letzterer ist ein kapazitiver mikro-elektro-mechanischer Senor und der Impulshammer ist mit einer Dehnmessstreifen basierten Ladungszelle ausgestattet.

Für die Umsetzung eines kostengünstigen Modalanalysesystems auf der Grundlage der zuvor genannten Komponenten, wird ein Mikrokontroller-basiertes Bus-System mit einem spezialisierten Kommunikationsprotokoll vorgeschlagen.