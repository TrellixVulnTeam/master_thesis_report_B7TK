% set counter to n-1:
\setcounter{chapter}{0}

\chapter{Introduction}

%-----------------------------------------------------
\section{Motivation}
The \ac{EMA} is a powerful tool for evaluating dynamic models of structures. Despite its extensive usage in the aerospace industry, in many other engineering fields the benefits of \ac{EMA} are overshadowed by the initial investment and the operator costs of an \ac{EMA} system. To enable \ac{MT} manufactures to test their products and validate their modal predictions. Progress in \ac{MEMS} technology enables the use of new generation of low-cost senors in \ac{EMA}.

%-----------------------------------------------------
\section{Related Work}

Considering the use of low-cost accelerometers in \ac{EMA} specifically, a two-point vibration measurement system with a bandwidth of \SI{500}{\hertz} has been developed~\cite{chan2017multiple}. The authors \citeauthor{chan2017multiple} used this system to conduct a multiple-point vibration tests on a \ac{MT}. Operating at lower frequencies, \citeauthor{beskhyroun2012low} used \ac{MEMS} based accelerometers to conduct an \ac{EMA} on building structures. Because we aim for higher bandwidths in this thesis, the transmission protocols used in the previously mentioned works do not satisfy the required data transmission rates. \citeauthor{piana2016experimental} developed a modal test system, which uses piezoelectric transducers that are typically used to tune musical instruments as response sensors~\cite{piana2016experimental}. Although giving an alternative option in the development of a low-cost system, this approach comes with two main drawbacks. Firstly, the sensors measure in one dimension only, compared two the now common three in accelerometer integrated circuits (\acs{IC}s). Because of this, the system is unsuited for some use cases. Secondly, when using piezoelectric transducers, one needs to conduct dedicated signal conditioning for each sensor. In comparison, this is already integrated in accelerometer \ac{IC}s that offer direct digital signal output. Hence more components must be used when going with the alternative option, increasing cost and form factor and decreasing reliability. Finally, a construction kit for a low-cost vibration analysis system was proposed by \citeauthor{vollmer2009construction} back in 2009~\cite{vollmer2009construction}. According to this paper, a broad product line of capacitive accelerometer \ac{IC}s, optimized for different bandwidths are available. But starting at a bandwith of \SI{2500}{\kilo\hertz} and higher only accelerometer \ac{IC}s with high acceleration ranges of \SI{\pm 70}{g} are available. For noise reduction, analog filters are sometimes partly or fully integrated. The noise in capacitive sensors can be expected to be in the range of \SIrange{0.1}{1}{\percent} of the measurement range, i.e.\ a factor of \SIrange{100}{1000}{\relax} worse than in piezoelectric sensors.

In the field of civil engineering, bridges and skyscrapers require continuous vibration signal logging for structural health monitoring. This leads to an increased interest in driving down the cost of accelerometer based vibration monitoring systems. \citeauthor{girolami2018modal} has developed a low-cost \ac{MEMS} systems for structural health monitoring of civil structures~\cite{girolami2018modal}. Typically, lower sensor bandwidths are required when analyzing building structures compared to \ac{MT}s. The data rate of the proposed monitoring system does not suffice for this thesis.

Structural health monitoring is also required in rotary systems such as gas and wind turbines. \citeauthor{esu2014integration} integrated low-cost accelerometers in wind turbines and logged data via radio frequency to a central device~\cite{esu2014integration}. An integration of digital accelerometer \ac{IC}s with sampling rates of up to \SI{1600}{\hertz} has been tested under laboratory conditions.

Addressing low-cost impact hammer constructions, \citeauthor{waltham2009construction} implemented a piezoelectric transducer in a hammer, that is designed to trigger barbecue lighters~\cite{waltham2009construction}. The hammer was tested by impact against wooden planks. It yielded robust signal in a wide range of forces. In contrast to this work, the load cell used in this thesis is strain gauge base in order to be more cost-effective compared to the typical piezoelectric sensors used in impact hammers. For inexpensive calibration of a load cell used for modal testing, \citeauthor{wang2015practical} introduced practical techniques~\cite{wang2015practical}.

%-----------------------------------------------------
\section{Overview}

First fundamentals in measurement instrumentation, sensors and \ac{EMA} is introduced in the state of the art. Then the \acf{DAC} developed for this project is presented in \autoref{chap:dac_software}. Testing environment are described in the \autoref{chap:test_setups} and measurement results are discussed in results. Finally the conclusion gives reflects on the project.
