% commands
\newcommand{\Adjoint}{\mbox{\rm Adj}}
\newcommand{\Area}{\mbox{\rm Area}}
\newcommand{\ACos}{{\mbox{\rm Cos}^{-1}}}
\newcommand{\ASin}{{\mbox{\rm Sin}^{-1}}}
\newcommand{\ATan}{{\mbox{\rm atan2}}}
\newcommand{\Code}[1]{{\tt #1}}
\newcommand{\Complex}{\mbox{\bf C}}
\newcommand{\Cross}{{\mbox{\rm Cross}}}
\newcommand{\Mydddot}[1]{\mbox{\shortstack{$.$\hspace*{-1pt}$.$\hspace*{-1pt}$.$\\$#1$}}}
\newcommand{\Degree}{\mbox{\rm degree}}
\newcommand{\Diag}{\mbox{\rm Diag}}
\newcommand{\Dim}{\mbox{\rm dim}}
\newcommand{\Dist}{\mbox{\rm Distance}}
\newcommand{\IntTwo}{\int\!\!\int}
\newcommand{\IntThree}{\int\!\!\int\! \!\int}
\newcommand{\Kernel}{\mbox{\rm kernel}}
\newcommand{\Kross}{\mbox{\rm Kross}}
\newcommand{\Grad}{\nabla}
\newcommand{\Perp}{\mbox{\rm Perp}}
\newcommand{\Point}[1]{{\cal #1}}
\newcommand{\Rank}{\mbox{\rm rank}}
\newcommand{\Range}{\mbox{\rm range}}
\newcommand{\Real}{{\mbox{\rm I}\hspace*{-2pt}\mbox{\rm R}}}
\newcommand{\RealSbt}{{\mbox{\rm\scriptsize I}\hspace*{-2pt}\mbox{\rm\scriptsize R}}}
% \newcommand{\Res}{\mbox{\rm resultant}}
\newcommand{\Sbt}[1]{{\mbox{\rm\scriptsize #1}}}
\newcommand{\MySign}{\mbox{\rm Sign}}
\newcommand{\SignSBT}{\mbox{\rm\scriptsize Sign}}
\newcommand{\Skew}{\mbox{\rm Skew}}
\newcommand{\Span}{\mbox{\rm Span}}
\newcommand{\SqrDist}{\mbox{\rm Distance$^2$}}
% \newcommand{\Trace}{\mbox{\rm Trace}}
\newcommand{\TRN}{{\mbox{\rm\scriptsize T}}}
\newcommand{\Vector}[1]{\mbox{\bf #1}}
\newcommand{\VectorM}[1]{\mbox{\boldmath $#1$}}
\newcommand{\Volume}{\mbox{\rm Volume}}

\newcommand{\IVec}{\mbox{\boldmath $\imath$}}
\newcommand{\JVec}{\mbox{\boldmath $\jmath$}}
\newcommand{\KVec}{\mbox{\boldmath $k$}}
\newcommand{\LVec}{\mbox{\boldmath $\ell$}}
\newcommand{\RMat}{{\cal R}}
\newcommand{\QMat}{{\cal Q}}
\newcommand{\QCMat}{\overline{\cal Q}}

\newcommand{\Lerp}{\mbox{\rm lerp}}
\newcommand{\Slerp}{\mbox{\rm slerp}}
\newcommand{\Quad}{\mbox{\rm quad}}
\newcommand{\Squad}{\mbox{\rm squad}}
\newcommand{\LT}[1]{\mathcal{L}\{#1\}}
\newcommand{\FT}[1]{\mathcal{F}\{#1\}}
\newcommand{\iu}{\mathrm{i}\mkren1mu}
\newcommand{\ju}{\mathrm{j}\mkern1mu}

\newcommand{\subsubsubsection}[1]{{\sc #1}}

\newcommand{\ODer}[2]{\frac{d #1}{d #2}}
\newcommand{\ODerT}[2]{\frac{d^2 #1}{d {#2}^2}}
\newcommand{\ODerM}[3]{\frac{d #1}{d #2 \, d #3}}
\newcommand{\PDer}[2]{\frac{\partial #1}{\partial #2}}
\newcommand{\PDerT}[2]{\frac{\partial^2 #1}{\partial {#2}^2}}
\newcommand{\PDerM}[3]{\frac{\partial^2 #1}{\partial #2 \, \partial #3}}

% requires asmath
\DeclareMathOperator\arcsinh{arcsinh}
\DeclareMathOperator\arccosh{arccosh}
\DeclareMathOperator\arctanh{arctanh}

%requires siunitx
\DeclareSIUnit\baud{Bd}


\newcommand{\estimates}{\mathrel{\widehat{=}}}% mass density symbol
\newcommand{\Den}{\delta}

% environments
\newenvironment{BArray}[1]{\left\{ \begin{array}{#1}}{\end{array} \right\}}
\newenvironment{Combin}{\left( \begin{array}{c}}{\end{array} \right)}
\newenvironment{Matrix}[1]{\left[ \begin{array}{#1}}{\end{array} \right]}

% "Figure" environment
\newtheorem{localFigure}{Figure}[chapter]
\newenvironment{Figure}[1]{
  \begin{center}
  \begin{minipage}{6in}
  \par\noindent\hspace*{0pt}\hrulefill

  \begin{localFigure} \label{#1}
}{
  \end{localFigure}
  \par\noindent\hspace*{0pt}\hrulefill
  \end{minipage}
  \end{center}
}

% "Table" environment
\newtheorem{localTable}{Table}[chapter]
\newenvironment{Table}[1]{
  \begin{center}
  \begin{minipage}{6in}

  \begin{localTable} \label{#1}
}{
  \end{localTable}
  \end{minipage}
  \end{center}
}

% "CDROM" environment for source code on disk
%\newenvironment{CDROM}[1]{
%  \label{#1}
%    \includegraphics{cdrom.png} \hspace*{0.1in}{\tt PointShop3D}. \rm
%}{
%  $\bowtie$
%}

% TO DO search symbol
\newcommand{\TODO}{\mbox{\large\bf TO DO}}
\newcommand{\REFR}{\mbox{\large\bf REFR}}

%  Terminates current page and paragraph, makes sure next page starts on
%  an odd-number, and generates a completely blank page, without page markers,
%  if necessary.
\newcommand{\clearemptydoublepage}{\newpage{\pagestyle{empty}\cleardoublepage}}

%%% Shoemake's commands
\DeclareMathOperator{\prp}{\text{\scshape perp}}
\DeclareMathOperator{\rot}{rot}
\DeclareMathOperator{\N}{N}
\providecommand{\vmag}[1]{\lVert#1\rVert}
\providecommand{\mutate}[1]{\overleftarrow{#1}}
\providecommand{\T}[1]{{#1}^{\mathrm T}}
% \newcommand{\cross}{\times}
\newcommand{\by}{\times}
\newcommand{\vect}[1]{\bm{#1}}
\newcommand{\mat}[1]{\bm{#1}} % or not
%\newcommand{\quat}[1]{\mathbf{#1}}
\newcommand{\quat}[1]{\ensuremath{\mathbf{\dot{#1}}}}
\newcommand{\vV}{\vect{v}}
\newcommand{\vU}{\vect{u}}
\newcommand{\vE}{\vect{e}}
\newcommand{\vUh}{\hat{\vect{u}}}
\newcommand{\mQ}{\mat{Q}}
\newcommand{\mR}{\mat{R}}
\newcommand{\mM}{\mat{M}}
\newcommand{\mA}{\mat{A}}
\newcommand{\mB}{\mat{B}}
\newcommand{\mI}{\mat{I}}
\newcommand{\mJ}{\mat{J}}
\newcommand{\mX}{\mat{X}}
\newcommand{\mY}{\mat{Y}}
\newcommand{\mZ}{\mat{Z}}
\newcommand{\qo}{\quat{1}}
\newcommand{\qi}{\quat{i}}
\newcommand{\qj}{\quat{j}}
\newcommand{\qk}{\quat{k}}
\newcommand{\xh}{{x}}
\newcommand{\yh}{{y}}
\newcommand{\zh}{{z}}
\newcommand{\ch}{c}
\newcommand{\sh}{s}
\newcommand{\gt}{\theta}


%%
%%
%%
